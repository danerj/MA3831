\documentclass{article}

\usepackage{amsmath}
\usepackage{amssymb}
\usepackage{amsfonts}
\usepackage{fullpage}
\usepackage[shortlabels]{enumitem}


\makeatletter
\def\moverlay{\mathpalette\mov@rlay}
\def\mov@rlay#1#2{\leavevmode\vtop{%
   \baselineskip\z@skip \lineskiplimit-\maxdimen
   \ialign{\hfil$\m@th#1##$\hfil\cr#2\crcr}}}
\newcommand{\charfusion}[3][\mathord]{
    #1{\ifx#1\mathop\vphantom{#2}\fi
        \mathpalette\mov@rlay{#2\cr#3}
      }
    \ifx#1\mathop\expandafter\displaylimits\fi}
\makeatother

\newcommand{\cupdot}{\charfusion[\mathbin]{\cup}{\cdot}}
\newcommand{\bigcupdot}{\charfusion[\mathop]{\bigcup}{\cdot}}

\title{MA 3831 Principles of Real Analysis 1 }
\author{Discussion 2}

\setlength\parindent{0pt}
\begin{document}
\maketitle


\subsection*{Exercise 1}

Show that the sequence converges using the definition of a convergent sequence.

\begin{enumerate}[a)]
\item $a_n = 3-\frac{4}{n}$

\item $b_n = \frac{2n+3}{3n+2}$
\end{enumerate} 

Answer:

\begin{enumerate}[a)]
\item It appears that $a_n$ converges to the limit $L = 3$. To prove this using the definition show that:\\

For any $\epsilon > 0$, there is an $N \in \mathbb{N}$ such that for any $n \in \mathbb{N}$ with $n > N$, $|a_n - 3| < \epsilon$. \\

Let $\epsilon > 0$. Choose $N$ to be an integer larger than $4/\epsilon$. For any integer $n > N$ it follows that $4/n < \epsilon$ so that

$$
|a_n - 3| = \biggr\lvert \left(3-\frac{4}{n}\right) - 3\biggr\rvert = \frac{4}{n} < \epsilon \;.
$$

Conclude that the sequence $a_n$ converges to 3.

\item Let $\epsilon > 0$. Choose $N \in \mathbb{N}$ such that $N > \frac{5}{9\epsilon}$. For any $n > N$, $\frac{5}{9n} < \epsilon$ so that

$$
\biggr\lvert b_n - \frac{2}{3} \biggr\lvert
= \biggr\lvert \frac{2n+3}{3n+2} - \frac{2}{3} \biggr\lvert
= \biggr\lvert \frac{5}{9n+6} \biggr\lvert
= \frac{5}{9n+6} 
< \frac{5}{9n}
< \epsilon .
$$
\end{enumerate}

\section*{Exercise 2}

Let $A = \{1-\frac{1}{n} : n \in \mathbb{N} \}$. Show using the definition of supremum that $\sup A = 1$. \\

Answer: Since $1/2 \in A$, $A \neq \emptyset$. Since $1-1/n < 1$ for any positive integer, $a \leq 1$ for all $a \in A$. Therefore $A$ is a nonempty set of real numbers bounded above. Thus $\sup A$ exists. To prove that $\sup A = 1$, show that:

\begin{enumerate}[(i)]
\item 1 is an upper bound of $A$
\item If $M \in \mathbb{R}$ is any upper bound of $A$ then $1 \leq M$.
\end{enumerate}

We have already established item (i). Note that item (ii) is equivalent (use the contrapositive) to the statement "Any real number strictly less than 1 is not an upper bound". Using more symbolic notation: $\forall \epsilon >0 $, $\exists$ $a \in A$ such that $1-\epsilon < a$ (if you show that such an $a$ exists you have shown that $1-\epsilon$ is not an upper bound of $A$). By the Archimedean property there is an integer $n_\epsilon > 1/\epsilon$. This implies $1-\epsilon < 1-1/n_\epsilon$. Since $1-1/n_\epsilon \in A$, $1-\epsilon$ is not an upper bound of $A$. Conclude $\sup A = 1$. 


\end{document}