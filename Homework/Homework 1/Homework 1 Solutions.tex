\documentclass{article}

\usepackage{amsmath}
\usepackage{amssymb}
\usepackage{amsfonts}
\usepackage{fullpage}
\usepackage[shortlabels]{enumitem}


\makeatletter
\def\moverlay{\mathpalette\mov@rlay}
\def\mov@rlay#1#2{\leavevmode\vtop{%
   \baselineskip\z@skip \lineskiplimit-\maxdimen
   \ialign{\hfil$\m@th#1##$\hfil\cr#2\crcr}}}
\newcommand{\charfusion}[3][\mathord]{
    #1{\ifx#1\mathop\vphantom{#2}\fi
        \mathpalette\mov@rlay{#2\cr#3}
      }
    \ifx#1\mathop\expandafter\displaylimits\fi}
\makeatother

\newcommand{\cupdot}{\charfusion[\mathbin]{\cup}{\cdot}}
\newcommand{\bigcupdot}{\charfusion[\mathop]{\bigcup}{\cdot}}

\title{MA 3831 Principles of Real Analysis Homework 1}
\author{Hubert J. Farnsworth}

\setlength\parindent{0pt}
\begin{document}
\maketitle


\subsection*{Exercise 1}
Show that for all $x,y \in \mathbb{Q}$,
\begin{enumerate}[a)]
\item
$\big\lvert \frac{1}{x} \big\rvert = \frac{1}{|x|}$ for $x \neq 0$. 

\item
$||x| - |y|| \leq |x-y| \leq |x| + |y|$ \\
\end{enumerate}

Answer:
\begin{enumerate}[a)]
\item
In class it was established that $|ab| = |a||b|$ for all $a,b \in \mathbb{Q}$. Set $a = x \neq 0$ and $b = 1/x$ in this identity to get:
$$
1 = |1| = \biggr\lvert x \frac{1}{x} \biggr\rvert = |x|\biggr\lvert \frac{1}{x} \biggr\rvert \iff \frac{1}{|x|} = \biggr\lvert \frac{1}{x} \biggr\rvert
$$

\item
In class the triangle identity was established: $|a+b| \leq |a| + |b|$ for all $a,b \in \mathbb{Q}$. Set $a = x$ and $b = -y$ in this identity to get: 
$$|x-y| = |x + (-y)| \leq |x| + |-y| = |x|+|y|$$

To establish $||x| - |y|| \leq |x-y|$ write, in turn, $x = x-y + y$ and $y = y-x+x$ to get:
\begin{align*}
|x| &= |(x-y) + y| \leq |x-y| + |y| \implies |x|-|y| \leq |x-y| \\
|y| &= |(y-x) + x| \leq |y-x| + |x| = |x-y| + |x| \implies -|x-y| \leq |x| - |y|
\end{align*}

Since $-|x-y| \leq |x|-|y| \leq |x-y|$, conclude $||x|-|y|| \leq |x-y|$.
\end{enumerate}

\subsection*{Exercise 2}
Let $T = (0,1) \cup \{2\}$. Find, with proof, $\sup T$. \\

Answer: Since $t \leq 2$ for all $t \in T$, 2 is an upper bound of the set $T$ of real numbers. Therefore $\sup T$ exists. To prove that $\sup T = 2$, show that for any $\epsilon > 0$, $2-\epsilon$ is not an upper bound of $T$: Let $\epsilon > 0$ be given. Since $2 \in T$ and $2-\epsilon < 2$, $2-\epsilon$ is not an upper bound of $T$. Since $2$ is an upper bound of $T$ and any real number less than $2$ is not an upper bound of $T$, conclude by the definition of supremum that $\sup T = 2$. 

\subsection*{Exercise 3}

Let $S$ and $T$ be two bounded above subsets of $\mathbb{R}$. Show that $S+T$ is bounded above, where $S+T$ is defined:
$$
S+T = \{s + t : s\in S, t \in T\} \;.
$$

Answer: Let $M_S$ be an upper bound of $S$ and $M_T$ be an upper bound of $T$. Then $S+T$ is bounded above by $M_S + M_T$. To prove this, suppose $u \in S+T$. By the definition of $S+T$ we know $u = s+t$ for some $s \in S$ and some $t \in T$. Since $s \in S$, $s \leq M_S$. Since $t \in T$, $t \leq M_T$. Thus $u = s+t \leq M_S + M_T$. Since $u$ was an arbitrary element of $S+T$, conclude $u \leq M_S + M_T$ for all $u \in S+T$ and therefore $S+T$ is bounded above.

\subsection*{Exercise 4}

(Exercise 1.4.4 of Understanding Analysis 2nd Edition by Stephen Abbott) Let $a<b$ be real numbers and consider the set $T = \mathbb{Q} \cap [a,b]$. Show $\sup T = b$. \\

Answer: For any $t \in T$, $t \in \mathbb{Q}$ and $t \in [a,b]$. Since $t \in [a,b]$, $t \leq b$ which means that $b$ is an upper bound of $T \subset \mathbb{R}$ and therefore $\sup T$ exists. To prove that $\sup T = b$, show that for any $\epsilon > 0$, $b - \epsilon$ is not an upper bound. Using the fact that between any two real numbers there exists a rational number, there exists a $t \in \mathbb{Q}$ such that $b-\epsilon < t < b$ since $b-\epsilon$ and $b$ are both real numbers. The rational number $t$ can be chosen such that $a\leq t$ as well (if $\epsilon > b-a$ one can simply create a subinterval of $(b-\epsilon, b)$ contained within $T$). Thus $t \in T$ with $b-\epsilon < t$ which shows that $b-\epsilon$ is not an upper bound of $T$. Conclude that since $b$ is an upper bound of $T$ and any real number less than $b$ is not an upper bound of $T$ that $\sup T = b$. 

\subsection*{Exercise 5}

Use the definition of a convergent sequence to show that $b_n = \frac{1}{\sqrt{n}}$ converges to $0$. \\

Answer: Let $\epsilon > 0$. By the Archimedean property of $\mathbb{R}$ there is an $N \in \mathbb{N}$ such that $N > 1/\epsilon^2$. This implies $\sqrt{N} = \sqrt{|N|} > |1/\epsilon| = 1/\epsilon$ which implies $0<1/\sqrt{N}<\epsilon$. For any $n \geq N$ we have $1/\sqrt{n} \leq 1/\sqrt{N}$. Therefore for any $n \geq N$:
$$
|b_n - 0| = \biggr\lvert \frac{1}{\sqrt{n}} - 0 \biggr\rvert = \frac{1}{\sqrt{n}} \leq \frac{1}{\sqrt{N}} < \epsilon \;.
$$

Since $\epsilon > 0$ was arbitrary, this shows that for any $\epsilon > 0$ there exists an $N = N(\epsilon) \in \mathbb{N}$ such that

$$|b_n - 0| < \epsilon \text{ for all } n\geq N \;.$$

Conclude by the definition of a convergent sequence that $b_n$ converges to $0$ as $n \rightarrow \infty$. 

\subsection*{Exercise 6}
Use the definition of a convergent sequence to show that any constant sequence is convergent. \\

Answer: Assume $b_n$ is a constant sequence with $b_n = b \in \mathbb{R}$ for $n = 1,2,3,\dots$. Then $b_n$ converges with limit $b$. To prove this, let $\epsilon > 0$ and take $N=1$. Then for any integer $n\geq N$, $b_n = b$ so that $|b_n - b| = |b-b| = |0| = 0 < \epsilon$. This shows that for any $\epsilon > 0$ one can find an $N \in \mathbb{N}$ such that $|b_n - b| < \epsilon$ for all $n \geq N$ from which we conclude by the definition of a convergent sequence that $b_n$ converges to $b$. Since $b_n$ was an arbitrary constant sequence, conclude that every constant sequence of real numbers is convergent. 
\end{document}